\begin{abstract}
In the current world of research data is a fundamental tool to disseminate scientific knowledge, to determine scholarship, and to provide credit and recognition to the authors of research endeavors. 
However, issues like data citation, handling and counting the credit generated by such citations are still open research questions. 

In this context, recently data credit has emerged as a new measure of value, built on top of the data citation theory. Data is a real value that represents the importance of data cited by a paper, or another research entity, in the context defined by that entity. As such, credit can be used to annotate data contained in curated scientific databases, and be considered as a measure for their importance. As such, it is a new method that, together with traditional citations, helps to recognize the value of data and its creators in a world more and more dependent on data. 

In this paper we explored the problem of Data Credit Distribution, the process by which credit is divided and assigned to the data in a database that are responsible for the production of data being cited by a research entity. 
In particular, we define two new distribution strategies, functions that perform this task, based on two form of data provenance, namely why-provenance, and how-provenance. 

As use case and for evaluation purposes, we adopt the IUPHAR/BPS Guide to Pharmacology (GtoPdb), a curated relational database.
We use these two strategies, together with a third one, based on lineage, previously defined in another our paper, to show how credit can highlight areas of a database that are frequently used, and how it can work as a new bibliometric measure for data and corresponding curators. Credit in particular rewards data and authors based on their research impact, and not merely on the number of citations.

Our experiments also highlight the fact that the why-provenance-based strategy is more sensible than lineage to the role of an input tuple in the generation of an output, rewarding more tuples that have a more fundamental role. Even more so with the how-provenance-based strategy. 
\end{abstract}