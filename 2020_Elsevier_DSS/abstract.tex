\begin{abstract}
Digital data is an important form of research product for which citation, and the generation of credit or recognition for authors, is still not well understood.
The notion of {\em data credit} has therefore recently emerged as a new metric, defined and based on data citation theory. 

Data credit is a real value that represents the importance of data cited by a paper or by another research entity. Credit can be used to annotate data contained in a curated scientific database, and then used as a measure for the importance and impact of that data in the research world. As such, it is a new method that, together with traditional citations, helps recognize the value of data and its creators.% in a world more and more dependent on data. 

In this paper we explore the problem of Data Credit Distribution, the process by which credit is distributed to the database parts   responsible for the production of data being cited by a research entity. 

We adopt as use case the IUPHAR/BPS Guide to Pharmacology (GtoPdb), a widely-used curated scientific relational database.
We define four new distribution strategies, the first two based on two forms of data provenance, why-provenance and how-provenance, the third based on the concept of responsibility, the fourth on the Shapley value. 

Using these distribution strategies we show how credit can highlight frequently used database areas and how it can be used as a new bibliometric measure for data and their corresponding curators. In particular, credit rewards data and authors based on their research impact, not merely on the number of citations.
We also show how these distribution strategies vary in their sensitivity to the role of an input tuple in the generation of the output data, and reward input tuples differently.


\eat{In the current world of research data is a fundamental method to disseminate scientific knowledge, to determine scholarship, and to provide credit and recognition to the authors of research endeavors. 
However, issues like data citation, handling and counting the credit generated by such citations are still open research questions. 

In this context, data credit has recently emerged as a new measure of value, defined and built on top of the data citation theory. Data credit is a real value that represents the importance of data cited by a paper, or by another research entity. As such, credit can be used to annotate data contained in curated scientific databases, and it can be considered as a measure for their importance and impact in the research world. As such, it is a new method that, together with traditional citations, helps to recognize the value of data and its creators in a world more and more dependent on data. 

In this paper we explore the problem of Data Credit Distribution, the process by which credit is divided and assigned to the data in a database that are responsible for the production of data being cited by a research entity. 

We adopt as use case the IUPHAR/BPS Guide to Pharmacology (GtoPdb), a curated and well-known scientific relational database.
We define two new distribution strategies, functions that perform this task, based on two form of data provenance, why-provenance, and how-provenance. 

Using different distribution strategies, we show how credit can highlight areas of a database that are frequently used, and how it can work as a new bibliometric measure for data and their corresponding curators. Credit in particular rewards data and authors based on their research impact, and not merely on the number of citations.
Also, we show how different distribution strategies, based on different types of data provenance, can be more sensible to the role of an input tuple in the generation of the output, and thus rewarding it differently.}
\end{abstract}


\begin{keyword}
Data Citation \sep Data Credit \sep Provenance \sep Causality and Responsibility \sep Shapley value  %% keywords here, in the form: keyword \sep keyword
%% MSC codes here, in the form: \MSC code \sep code
%% or \MSC[2008] code \sep code (2000 is the default)

\end{keyword}