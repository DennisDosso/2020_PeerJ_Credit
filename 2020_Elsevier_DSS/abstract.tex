\begin{abstract}
In the current world of research data is a fundamental method to disseminate scientific knowledge, to determine scholarship, and to provide credit and recognition to the authors of research endeavors. 
However, issues like data citation, handling and counting the credit generated by such citations are still open research questions. 

In this context, data credit has recently emerged as a new measure of value, defined and built on top of the data citation theory. Data credit is a real value that represents the importance of data cited by a paper, or by another research entity. As such, credit can be used to annotate data contained in curated scientific databases, and it can be considered as a measure for their importance and impact in the research world. As such, it is a new method that, together with traditional citations, helps to recognize the value of data and its creators in a world more and more dependent on data. 

In this paper we explore the problem of Data Credit Distribution, the process by which credit is divided and assigned to the data in a database that are responsible for the production of data being cited by a research entity. 

We adopt as use case the IUPHAR/BPS Guide to Pharmacology (GtoPdb), a curated and well-known scientific relational database.
We define two new distribution strategies, functions that perform this task, based on two form of data provenance, why-provenance, and how-provenance. 

Using different distribution strategies, we show how credit can highlight areas of a database that are frequently used, and how it can work as a new bibliometric measure for data and their corresponding curators. Credit in particular rewards data and authors based on their research impact, and not merely on the number of citations.
Also, we show how different distribution strategies, based on different types of data provenance, can be more sensible to the role of an input tuple in the generation of the output, and thus rewarding it differently.
\end{abstract}


\begin{keyword}
Data Citation \sep Data Credit
%% keywords here, in the form: keyword \sep keyword
%% MSC codes here, in the form: \MSC code \sep code
%% or \MSC[2008] code \sep code (2000 is the default)

\end{keyword}