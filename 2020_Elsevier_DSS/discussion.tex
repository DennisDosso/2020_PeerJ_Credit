\section{Discussion}
\label{sec:discussion}

\rone{We note that, in our experiments, we always assumed that the credit carried by an output tuple is 1. Thus, each tuple in the output has equal importance. This in general may not be true, since different tuples in the output may have different weight, depending on the context of the citation. For example, data that is fundamental for the results of a paper may have more credit than data being cited as a reference. 
\emph{Credit generation}, i.e. the process by which the credit of the output tuples is decided, is research problem with its own dignity and complexities, and we did not face it in this paper.}

\rone{From the point of view of the model, even when the credit of the output tuples is different than 1, nothing needs to change in the models presented here, since they were defined for a generic value $k$. We note that, if the quantity of credit carried by an output tuple changes, as a consequence the final distribution will change, since certain tuples will be more ``impactful'' (i.e., distribute more credit) than others. With different quantities of credit, therefore, new results, different from the ones obtained in the previous sections, may be found. These results will depend on the nature of the context and the quantity of credit being considered. }